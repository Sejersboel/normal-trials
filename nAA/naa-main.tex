\subsection*{Intro og trash}
Aetherian Archive er en af de 3 originale trials i Craglorn, der alle er en del
af basis spillet. De er alle forholdsvis nemme, hurtige og på normal
``tilgivende'', hvis man ikke forholder sig til mechanics. Det er lidt en anden
historie på veteran. Dog er Atherian Archive en god begynder trial, da der er
en del mechanics, hvor spillerne skal holde øje med omgivelser, bossers
bevægelser og røde cirkler på jorden. Dette kan være god træning til noget af
det lidt sværere.\\
\\
I denne trial skal man bruge 1 tank, 2 healere og 9 dd'er.\\
Inden man stacker på døren, deles gruppen i 3 grupper af 4. 1 healer stiller
sig på hvert bord (højre og venstre) og tanken i midten. DD'er fordeler sig i
de tre grupper, helst med magicka-dd i højre gruppe, men det gør ikke så meget
på normal. Folk skal gerne huske deres gruppe hhv. højre, venstre og midt.
\subsubsection*{Elite Adds}
\begin{itemize}
  \item \textbf{(Firstmage) Overcharger} mage med 2 flag på ryggen.
    Overchargers har altid 1. prioritet. De har to abilities, man skal være
    opmærksom på: 
    \begin{itemize}
      \item \emph{Overcharge}, som smides på en tilfældig spiller. Dette
        giver spilleren en forholdsvis stor rød cirkel under fødderne, der følger
        spilleren og skader hvert sekund. Det skader også andre, der står i den
        røde cirkel. Healerne kan godt følge med, hvis folk står i én
        overcharge, men hvis to personer med overcharge overlapper, kan det
        være svært.
      \item \emph{Call Lightning} rammer en tilfældig spiller, og laver
        gentagne gange små røde cirkler, hvor man står. Disse skal man træde ud 
        af for ikke at tage skade.
    \end{itemize}
  \item \textbf{(Firstmage) Chainspinner} dual wield med 1 flag på ryggen.
    Disse adds har 2. prioritet, da de laver nogle AOE'er, der skader og
    CC'er. 
  \item \textbf{(Firstmage) Nullifier} Healer med 3 flag på ryggen. Disse er
    rare at få ned hurtigt, da de healer de andre, men de har 3. prioritet, da
    man 'bare' kan interrupt'e deres heals.
\end{itemize}
Udover disse adds er der nogle almindelige adds, som bare skal dø til AOE og,
lige til at starte med, nogle fire og frost attronachs, som bare er som de
plejer. 
\subsection*{Første Adds}
\subsubsection*{Fire Attronachs}
Når man starter, ved at gå gennem døren og løbe op ad trappen, er det vigtigt
at alle løber HELT op ad og lidt væk fra trappen, da der kommer meget ild på
selve trappen og lige på toppen af den. Ovenfor den første trappe står der
nogle Fire Attronachs, som tanken bare skal samle, og alle andre slå ned. Vær
opmærksom på, at de springer i luften, efter de dør.
\subsubsection*{Frost Attronachs}
Når de er døde, kan gruppen gå gennem døren og gennem de frosne bibliotek, hvor
man skal forsøge at komme udenom de frysende hvirvelvinde, der flyver rundt
over gulvet. Bliver man fanget, skal man rulle, for at komme videre. Alle mødes
ved døren, på den anden side. Når hele gruppen er klar, går man gennem døren,
og tanken vender de to store Frost Attronachs om, da de har et stort cleave. De
slås ned, og man kan forsætte op ad trappen, hvor alle kan se den gule
beskyttende cirkel, som man skal være opmærksom på ved den første boss.
\subsubsection*{Flag-Adds}
Når man kommer gennem døren, samles HELE gruppen lige inde i det næste rum. For
at gøre det nemt for tanken(og i virkeligheden hele gruppen), er det vigtigt at
HELE gruppen løber SAMLET op mod add-gruppen og at alle løber helt op til
gruppen. Hvis alle gør dette, bliver add-gruppen stående, og man kan nemt slå dem
ned med destro-ult, elemental ring, jabs/sweeps, scythe osv. Hver opmærksom på,
at der er en Overcharger, som naturligvis har størst prioritet.

\subsection*{Boss 1: Lightning Storm Attronach}
I denne boss-kamp tankes bossen, hvor den står, i midten. DD'er og healere
stacker bag bossen. I løbet af kampen holdes øje med, hvor den gule cirkel
kommer. Da det ikke altid er alle, der kan se cirklen, er det ALLES opgave at
holde øje med, hvor den kommer. Den kan komme i en af 5 positioner:
\begin{itemize}
  \item Indgang
  \item Udgang
  \item Venstre
  \item Højre for
  \item Højre bag
\end{itemize}
Når cirklen kommer, kaldes det straks ud, og ALLE i gruppen ruller i retning af
cirklen og sprinter resten af vejen, mens healere gerne skal smide
\emph{Illustrious Healing} eller \emph{Energy Orb} mv. på vejen. Når
lyn-effekten stopper, skal tanken skynder sig tilbage til bossen, så den ikke
flytter sig. Dette kan godt ske lidt før, den gule cirkel forsvinder. Alle
andre bliver stående, til cirklen forsvinder, hvorefter de løber tilbage bag
bossen og fortsætter. Som udgangspunk skal man ALTID løbe til cirklen. Også
selv om bossen er på 1\%.
\subsection*{Adds 2}

