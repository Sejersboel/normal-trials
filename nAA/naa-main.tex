\subsection*{Intro og trash}
Aetherian Archive er en af de 3 originale trials i Craglorn, der alle er en del
af basis spillet. De er alle forholdsvis nemme, hurtige og på normal
``tilgivende'', hvis man ikke forholder sig til mechanics. Det er lidt en anden
historie på veteran. Dog er Atherian Archive en god begynder trial, da der er
en del mechanics, hvor spillerne skal holde øje med omgivelser, bossers
bevægelser og røde cirkler på jorden. Dette kan være god træning til noget af
det lidt sværere.\\
\\
I denne trial skal man bruge 1 tank, 2 healere og 9 dd'er.\\
Inden man stacker på døren, deles gruppen i 3 grupper af 4. 1 healer stiller
sig på hvert bord (højre og venstre) og tanken i midten. DD'er fordeler sig i
de tre grupper, helst med magicka-dd i højre gruppe, men det gør ikke så meget
på normal. Folk skal gerne huske deres gruppe hhv. højre, venstre og midt.
\subsubsection*{Elite Adds}
\begin{itemize}
  \item \textbf{(Firstmage) Overcharger} mage med 2 flag på ryggen.
    Overchargers har altid 1. prioritet. De har to abilities, man skal være
    opmærksom på: 
    \begin{itemize}
      \item \emph{Overcharge}, som smides på en tilfældig spiller. Dette
        giver spilleren en forholdsvis stor rød cirkel under fødderne, der følger
        spilleren og skader hvert sekund. Det skader også andre, der står i den
        røde cirkel. Healerne kan godt følge med, hvis folk står i én
        overcharge, men hvis to personer med overcharge overlapper, kan det
        være svært.
      \item \emph{Call Lightning} rammer en tilfældig spiller, og laver
        gentagne gange små røde cirkler, hvor man står. Disse skal man træde ud 
        af for ikke at tage skade.
    \end{itemize}
  \item \textbf{(Firstmage) Chainspinner} dual wield med 1 flag på ryggen.
    Disse adds har 2. prioritet, da de laver nogle AOE'er, der skader og
    CC'er. 
  \item \textbf{(Firstmage) Nullifier} Healer med 3 flag på ryggen. Disse er
    rare at få ned hurtigt, da de healer de andre, men de har 3. prioritet, da
    man 'bare' kan interrupt'e deres heals.
\end{itemize}
Udover disse adds er der nogle almindelige adds, som bare skal dø til AOE og,
lige til at starte med, nogle fire og frost attronachs, som bare er som de
plejer. 
\subsection*{Første Adds}
\subsubsection*{Fire Attronachs}
Når man starter, ved at gå gennem døren og løbe op ad trappen, er det vigtigt
at alle løber HELT op ad og lidt væk fra trappen, da der kommer meget ild på
selve trappen og lige på toppen af den. Ovenfor den første trappe står der
nogle Fire Attronachs, som tanken bare skal samle, og alle andre slå ned. Vær
opmærksom på, at de springer i luften, efter de dør.
\subsubsection*{Frost Attronachs}
Når de er døde, kan gruppen gå gennem døren og gennem de frosne bibliotek, hvor
man skal forsøge at komme udenom de frysende hvirvelvinde, der flyver rundt
over gulvet. Bliver man fanget, skal man rulle, for at komme videre. Alle mødes
ved døren, på den anden side. Når hele gruppen er klar, går man gennem døren,
og tanken vender de to store Frost Attronachs om, da de har et stort cleave. De
slås ned, og man kan forsætte op ad trappen, hvor alle kan se den gule
beskyttende cirkel, som man skal være opmærksom på ved den første boss.
\subsubsection*{Flag-Adds}
Når man kommer gennem døren, samles HELE gruppen lige inde i det næste rum. For
at gøre det nemt for tanken(og i virkeligheden hele gruppen), er det vigtigt at
HELE gruppen løber SAMLET op mod add-gruppen og at alle løber helt op til
gruppen. Hvis alle gør dette, bliver add-gruppen stående, og man kan nemt slå dem
ned med destro-ult, elemental ring, jabs/sweeps, scythe osv. Hver opmærksom på,
at der er en Overcharger, som naturligvis har størst prioritet.

\subsection*{Boss 1: Lightning Storm Attronach}
I denne boss-kamp tankes bossen, hvor den står, i midten. DD'er og healere
stacker bag bossen. I løbet af kampen holdes øje med, hvor den gule cirkel
kommer. Da det ikke altid er alle, der kan se cirklen, er det ALLES opgave at
holde øje med, hvor den kommer. Den kan komme i en af 5 positioner:
\begin{itemize}
  \item Indgang
  \item Udgang
  \item Venstre
  \item Højre for
  \item Højre bag
\end{itemize}
Når cirklen kommer, kaldes det straks ud, og ALLE i gruppen ruller i retning af
cirklen og sprinter resten af vejen, mens healere gerne skal smide
\emph{Illustrious Healing} eller \emph{Energy Orb} mv. på vejen. Når
lyn-effekten stopper, skal tanken skynder sig tilbage til bossen, så den ikke
flytter sig. Dette kan godt ske lidt før, den gule cirkel forsvinder. Alle
andre bliver stående, til cirklen forsvinder, hvorefter de løber tilbage bag
bossen og fortsætter. Som udgangspunk skal man ALTID løbe til cirklen. Også
selv om bossen er på 1\%.
\subsection*{Adds 2}
Efter den første boss, splitter gruppen op 1 de tre grupper, som blev delv
tidligere. Hver spiller har hver sin lille platform at stå på. Gruppen i midten
kan \emph{cheese} mechancis en smule ved at sneake inden de går op på deres
platform. 
Når gruppen i midten er teleporteret, kan de starte med at sneake en smule
baglæns. På den måde undgår de en kamp, hvor adds bliver ved med at komme
igen.\\
Gruppen til højre og venstre har hver deres adds at slå ihjel. Når man er
færdig i sin side, siger man hhv. hørje og venstre færdig og løber ad den
bageste bro over til den anden side og hjælper sine kammerater dér. Når begge
sider er færdig, slå man adds i midten ned og alle stiller sig på en platform i
midten, hvorefter alle kommer tilbage til der, hvor den første bosskamp fandt
sted.\\
Når alle er kommet tilbage, formes en bro som alle løber over.
Når man kommer over broen, er der en sten lige fremme, som alle stiller sig på.
Her er det næst add pull, som kommer lidt i bølger. Der kommer flere af både
Overchargers, Chainspinners og Nullifiers.\\
Når alle adds er døde, stille gruppen sig hen på hver deres lille platform, og
man bliver teleporteret ned til den næste boss.
\subsection*{Boss 2: Foundation Stone Attronach}
Denne stone attronach tankes, hvor den står, og alle stacker på tanken. Når
bossen kommer ned på 75\% liv, begynder den at klaske sammen og kaste sten ud
mod tilfældige spillere. Når den klasker sammen skal \emph{ALLE UD} og
soft-stacke og block caste. Når bossen rejser sig op igen, skal \emph{ALLE IND}
og stacke på tanken igen. Tanken flytter sig ikke, under \emph{ALLE UD/ALLE
IND}. Dette fortsætter resten af kampen. Adds ignoreres.\\
Når man soft-stacker, er det vigtigt, at man ikke løber for langt væk. Hvis man
gør det, kan healerne ikke nå en, og man risikerer at dø. 
\subsection*{Adds 3}
Når bossen er død, løber alle op ad broen, der danner sig bag ved bossen. Her
deles holdet op i de tre grupper igen. Denne gang kan grupperne ikke hjælpe
hinanden, så det er vigtigt at få resset op, hvis der er nogen, der dør.
I gruppen i midten, tager tanken en enkelt add til side og resten af
midtergruppen dræber de resterende adds.
Igen bliver der kaldt hhv. højre og venstre færdig. Når begge sider er færdig,
dræbes den sidste add, tanken står med. Så vil der blive dannet en bro videre
for alle tre grupper og alle løber videre til den næste boss.
\subsection*{Boss 3: Varlariel}
Varlariel er en whispmother, der som alle andre whispmothers deler sig.\\
Tanken står med hende og holder taunt og resten af gruppen soft-stacker omkring
hende og dps-danser en smule, da der kommer lidt ned på hver spiller, som man
ikke skal dele med andre.
Når Varlariel deler sig, dræbes alle hendes kloner, pånær den ved 'juletræet'.
Hvis ikke de dræbes, sker der en eksplosion, så skader $x \cdot
resterenden\_kloner$. Grunden til at man efterlader den ved juletræet er, at
man godt kan overleve en(Selv på veteran), og ens tid er bedre brugt på at
skade Varlariel, da der kommer flere kloner for hver gang hun deler sig.\\
\\
Når Varlariel er død, løber alle videre op og stiller sig på hver deres lille
teleportations-platform, hvorefter holdet bliver taget til den sidste arena.
\subsection*{Adds 4}
Der står allerede nogle adds klar, når man ankommer til den sidste arena. Når
de er døde, kommer der et par bølger mere, inden bosskampen starter. Vær
opmærksom på, at der er både to og tre Overchargers i bølgerne. Tanken kan
forsøge at stacke disse med Line-of-sight. Når alle adds er døde, starter
bosskampen.
\subsection*{Boss 4: The Mage}
Bossen kommer ned og stiller sig i midten af arenaen. Alle soft-stacker et
stykke fra hendem medmindre man har behov for at stå i melee-afstand.\\
Hvis man
har behov for at stå i melee-afstand, aftales det inden man porter til arenaen.
Dem, der har behov for melee, står to og to sammen til hhv. højre, venstre,
bagved og foran bossen, og fylder ud i den rækkefølge. Hvis der er et ulige
antal, kan der i den sidste position, der fyldes ud, stå en alene. De andre
position STACKER helt oven i hinanden. Hvis man ikke gør det på den måde,
kommer der chain-lightning gennem gruppen, som skader mere og mere for hver
person, det rammer. \\
Tanken står på stenen, man kom ind på, og taunter de økser, der kommer som
kampen skrider frem. En healer har til formål at heale og give ressourcer til
tanken. Den anden healer healer gruppen.\\
\subsubsection*{Miner}
I løbet at kampen kommer der små røde cirkler med en hvis plet i midten. Disse
skal dd'er blocke og gå ind i. Hvis ikke dette gøres, bliver de til sorte
huller, der skader meget og kan wipe gruppen, hvis de tager overhånd.
\subsubsection*{Mini mages}
Disse små kloner af selve bossen kommer med jævne mellemrum. Disse skal have
fokus fra alle, der kan nå dem fra hvor man står. De skal helst ned så hurtigt
som muligt. 
\subsubsection*{Execute!}
Omkring 15\% liv, slår bossen alle om på ryggen tre gange. Efter tredje gang,
løber alle ind og executer bossen. Herfra er det et dps-race. Det burde ikke
være et problem på normal, men man kan godt wipe, hvis det tager for lang tid.
\\
\\
Når bossen er død, skal man gennem portalen for at aflevere questen.
